%% Generated by Sphinx.
\def\sphinxdocclass{report}
\documentclass[letterpaper,10pt,english]{sphinxmanual}
\ifdefined\pdfpxdimen
   \let\sphinxpxdimen\pdfpxdimen\else\newdimen\sphinxpxdimen
\fi \sphinxpxdimen=.75bp\relax

\PassOptionsToPackage{warn}{textcomp}
\usepackage[utf8]{inputenc}
\ifdefined\DeclareUnicodeCharacter
 \ifdefined\DeclareUnicodeCharacterAsOptional
  \DeclareUnicodeCharacter{"00A0}{\nobreakspace}
  \DeclareUnicodeCharacter{"2500}{\sphinxunichar{2500}}
  \DeclareUnicodeCharacter{"2502}{\sphinxunichar{2502}}
  \DeclareUnicodeCharacter{"2514}{\sphinxunichar{2514}}
  \DeclareUnicodeCharacter{"251C}{\sphinxunichar{251C}}
  \DeclareUnicodeCharacter{"2572}{\textbackslash}
 \else
  \DeclareUnicodeCharacter{00A0}{\nobreakspace}
  \DeclareUnicodeCharacter{2500}{\sphinxunichar{2500}}
  \DeclareUnicodeCharacter{2502}{\sphinxunichar{2502}}
  \DeclareUnicodeCharacter{2514}{\sphinxunichar{2514}}
  \DeclareUnicodeCharacter{251C}{\sphinxunichar{251C}}
  \DeclareUnicodeCharacter{2572}{\textbackslash}
 \fi
\fi
\usepackage{cmap}
\usepackage[T1]{fontenc}
\usepackage{amsmath,amssymb,amstext}
\usepackage{babel}
\usepackage{times}
\usepackage[Bjarne]{fncychap}
\usepackage{sphinx}

\usepackage{geometry}

% Include hyperref last.
\usepackage{hyperref}
% Fix anchor placement for figures with captions.
\usepackage{hypcap}% it must be loaded after hyperref.
% Set up styles of URL: it should be placed after hyperref.
\urlstyle{same}

\addto\captionsenglish{\renewcommand{\figurename}{Fig.}}
\addto\captionsenglish{\renewcommand{\tablename}{Table}}
\addto\captionsenglish{\renewcommand{\literalblockname}{Listing}}

\addto\captionsenglish{\renewcommand{\literalblockcontinuedname}{continued from previous page}}
\addto\captionsenglish{\renewcommand{\literalblockcontinuesname}{continues on next page}}

\addto\extrasenglish{\def\pageautorefname{page}}

\setcounter{tocdepth}{1}



\title{pyEMU Documentation}
\date{Jun 28, 2018}
\release{0.3}
\author{Jeremy White, Mike Fienen, John Doherty}
\newcommand{\sphinxlogo}{\vbox{}}
\renewcommand{\releasename}{Release}
\makeindex

\begin{document}

\maketitle
\sphinxtableofcontents
\phantomsection\label{\detokenize{index::doc}}


pyEMU \phantomsection\label{\detokenize{index:id1}}{\hyperref[\detokenize{index:wfd16}]{\sphinxcrossref{{[}WFD16{]}}}} is a set of python modules for performing linear and non-linear
uncertainty analysis including parameter and forecast analyses, data-worth
analysis, and error-variance analysis. These python modules can interact with
the PEST \phantomsection\label{\detokenize{index:id2}}{\hyperref[\detokenize{index:doh15}]{\sphinxcrossref{{[}DOH15{]}}}} and PEST++ \phantomsection\label{\detokenize{index:id3}}{\hyperref[\detokenize{index:wwhd15}]{\sphinxcrossref{{[}WWHD15{]}}}}  suites and use terminology consistent
with them.   pyEMU is written in an object-oriented programming style, and
thus expects that users will write, or adapt, client code in python to
implement desired analysis.  {\hyperref[\detokenize{source/oop::doc}]{\sphinxcrossref{\DUrole{doc}{Notes on Object-Oriented Programming}}}} are provided in this documentation.

pyEMU is available via \sphinxhref{https://github.com/jtwhite79/pyemu}{github} .


\chapter{Contents}
\label{\detokenize{index:contents}}\label{\detokenize{index:notes}}

\section{Notes on Object-Oriented Programming}
\label{\detokenize{source/oop:notes-on-object-oriented-programming}}\label{\detokenize{source/oop::doc}}
blah, blah, blah


\section{Glossary}
\label{\detokenize{source/glossary:glossary}}\label{\detokenize{source/glossary::doc}}\begin{description}
\item[{class\index{class|textbf}\phantomsection\label{\detokenize{source/glossary:term-class}}}] \leavevmode
blah, blah, blah

\item[{object\index{object|textbf}\phantomsection\label{\detokenize{source/glossary:term-object}}}] \leavevmode\item[{instance\index{instance|textbf}\phantomsection\label{\detokenize{source/glossary:term-instance}}}] \leavevmode
generated from the class…

\end{description}


\chapter{Technical Documentation}
\label{\detokenize{index:technical-documentation}}\begin{itemize}
\item {} 
\DUrole{xref,std,std-ref}{genindex}

\item {} 
\DUrole{xref,std,std-ref}{modindex}

\item {} 
\DUrole{xref,std,std-ref}{search}

\end{itemize}


\chapter{References}
\label{\detokenize{index:references}}
\begin{sphinxthebibliography}{WWHD15}
\bibitem[DOH15]{\detokenize{DOH15}}{\phantomsection\label{\detokenize{index:doh15}} 
Doherty, J., 2015. Calibration and
Uncertainty Analysis for Complex Environmental Models:  Brisbane, Australia, Watermark Numerical
Computing, \sphinxurl{http://www.pesthomepage.org/Home.php} .
}
\bibitem[WFD16]{\detokenize{WFD16}}{\phantomsection\label{\detokenize{index:wfd16}} 
White, J.T., Fienen, M.N., and Doherty, J.E., 2016, A python framework
for environmental model uncertainty analysis:  Environmental Modeling \&
Software, v. 85, pg. 217-228, \sphinxurl{https://doi.org/10.1016/j.envsoft.2016.08.017} .
}
\bibitem[WWHD15]{\detokenize{WWHD15}}{\phantomsection\label{\detokenize{index:wwhd15}} 
Welter, D.E., White, J.T., Hunt, R.J., and Doherty, J.E., 2015,
Approaches in highly parameterized inversion: PEST++ Version 3, a Parameter
ESTimation and uncertainty analysis software suite optimized for large
environmental models: U.S. Geological Survey Techniques and Methods, book 7,
section C12, 54 p., \sphinxurl{https://doi.org/10.3133/tm7C12} .
}
\end{sphinxthebibliography}



\renewcommand{\indexname}{Index}
\printindex
\end{document}