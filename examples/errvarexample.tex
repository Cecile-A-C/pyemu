
% Default to the notebook output style

    


% Inherit from the specified cell style.




    
\documentclass{article}

    
    
    \usepackage{graphicx} % Used to insert images
    \usepackage{adjustbox} % Used to constrain images to a maximum size 
    \usepackage{color} % Allow colors to be defined
    \usepackage{enumerate} % Needed for markdown enumerations to work
    \usepackage{geometry} % Used to adjust the document margins
    \usepackage{amsmath} % Equations
    \usepackage{amssymb} % Equations
    \usepackage{eurosym} % defines \euro
    \usepackage[mathletters]{ucs} % Extended unicode (utf-8) support
    \usepackage[utf8x]{inputenc} % Allow utf-8 characters in the tex document
    \usepackage{fancyvrb} % verbatim replacement that allows latex
    \usepackage{grffile} % extends the file name processing of package graphics 
                         % to support a larger range 
    % The hyperref package gives us a pdf with properly built
    % internal navigation ('pdf bookmarks' for the table of contents,
    % internal cross-reference links, web links for URLs, etc.)
    \usepackage{hyperref}
    \usepackage{longtable} % longtable support required by pandoc >1.10
    \usepackage{booktabs}  % table support for pandoc > 1.12.2
    

    
    
    \definecolor{orange}{cmyk}{0,0.4,0.8,0.2}
    \definecolor{darkorange}{rgb}{.71,0.21,0.01}
    \definecolor{darkgreen}{rgb}{.12,.54,.11}
    \definecolor{myteal}{rgb}{.26, .44, .56}
    \definecolor{gray}{gray}{0.45}
    \definecolor{lightgray}{gray}{.95}
    \definecolor{mediumgray}{gray}{.8}
    \definecolor{inputbackground}{rgb}{.95, .95, .85}
    \definecolor{outputbackground}{rgb}{.95, .95, .95}
    \definecolor{traceback}{rgb}{1, .95, .95}
    % ansi colors
    \definecolor{red}{rgb}{.6,0,0}
    \definecolor{green}{rgb}{0,.65,0}
    \definecolor{brown}{rgb}{0.6,0.6,0}
    \definecolor{blue}{rgb}{0,.145,.698}
    \definecolor{purple}{rgb}{.698,.145,.698}
    \definecolor{cyan}{rgb}{0,.698,.698}
    \definecolor{lightgray}{gray}{0.5}
    
    % bright ansi colors
    \definecolor{darkgray}{gray}{0.25}
    \definecolor{lightred}{rgb}{1.0,0.39,0.28}
    \definecolor{lightgreen}{rgb}{0.48,0.99,0.0}
    \definecolor{lightblue}{rgb}{0.53,0.81,0.92}
    \definecolor{lightpurple}{rgb}{0.87,0.63,0.87}
    \definecolor{lightcyan}{rgb}{0.5,1.0,0.83}
    
    % commands and environments needed by pandoc snippets
    % extracted from the output of `pandoc -s`
    \DefineVerbatimEnvironment{Highlighting}{Verbatim}{commandchars=\\\{\}}
    % Add ',fontsize=\small' for more characters per line
    \newenvironment{Shaded}{}{}
    \newcommand{\KeywordTok}[1]{\textcolor[rgb]{0.00,0.44,0.13}{\textbf{{#1}}}}
    \newcommand{\DataTypeTok}[1]{\textcolor[rgb]{0.56,0.13,0.00}{{#1}}}
    \newcommand{\DecValTok}[1]{\textcolor[rgb]{0.25,0.63,0.44}{{#1}}}
    \newcommand{\BaseNTok}[1]{\textcolor[rgb]{0.25,0.63,0.44}{{#1}}}
    \newcommand{\FloatTok}[1]{\textcolor[rgb]{0.25,0.63,0.44}{{#1}}}
    \newcommand{\CharTok}[1]{\textcolor[rgb]{0.25,0.44,0.63}{{#1}}}
    \newcommand{\StringTok}[1]{\textcolor[rgb]{0.25,0.44,0.63}{{#1}}}
    \newcommand{\CommentTok}[1]{\textcolor[rgb]{0.38,0.63,0.69}{\textit{{#1}}}}
    \newcommand{\OtherTok}[1]{\textcolor[rgb]{0.00,0.44,0.13}{{#1}}}
    \newcommand{\AlertTok}[1]{\textcolor[rgb]{1.00,0.00,0.00}{\textbf{{#1}}}}
    \newcommand{\FunctionTok}[1]{\textcolor[rgb]{0.02,0.16,0.49}{{#1}}}
    \newcommand{\RegionMarkerTok}[1]{{#1}}
    \newcommand{\ErrorTok}[1]{\textcolor[rgb]{1.00,0.00,0.00}{\textbf{{#1}}}}
    \newcommand{\NormalTok}[1]{{#1}}
    
    % Define a nice break command that doesn't care if a line doesn't already
    % exist.
    \def\br{\hspace*{\fill} \\* }
    % Math Jax compatability definitions
    \def\gt{>}
    \def\lt{<}
    % Document parameters
    \title{errvarexample}
    
    
    

    % Pygments definitions
    
\makeatletter
\def\PY@reset{\let\PY@it=\relax \let\PY@bf=\relax%
    \let\PY@ul=\relax \let\PY@tc=\relax%
    \let\PY@bc=\relax \let\PY@ff=\relax}
\def\PY@tok#1{\csname PY@tok@#1\endcsname}
\def\PY@toks#1+{\ifx\relax#1\empty\else%
    \PY@tok{#1}\expandafter\PY@toks\fi}
\def\PY@do#1{\PY@bc{\PY@tc{\PY@ul{%
    \PY@it{\PY@bf{\PY@ff{#1}}}}}}}
\def\PY#1#2{\PY@reset\PY@toks#1+\relax+\PY@do{#2}}

\expandafter\def\csname PY@tok@gd\endcsname{\def\PY@tc##1{\textcolor[rgb]{0.63,0.00,0.00}{##1}}}
\expandafter\def\csname PY@tok@gu\endcsname{\let\PY@bf=\textbf\def\PY@tc##1{\textcolor[rgb]{0.50,0.00,0.50}{##1}}}
\expandafter\def\csname PY@tok@gt\endcsname{\def\PY@tc##1{\textcolor[rgb]{0.00,0.27,0.87}{##1}}}
\expandafter\def\csname PY@tok@gs\endcsname{\let\PY@bf=\textbf}
\expandafter\def\csname PY@tok@gr\endcsname{\def\PY@tc##1{\textcolor[rgb]{1.00,0.00,0.00}{##1}}}
\expandafter\def\csname PY@tok@cm\endcsname{\let\PY@it=\textit\def\PY@tc##1{\textcolor[rgb]{0.25,0.50,0.50}{##1}}}
\expandafter\def\csname PY@tok@vg\endcsname{\def\PY@tc##1{\textcolor[rgb]{0.10,0.09,0.49}{##1}}}
\expandafter\def\csname PY@tok@m\endcsname{\def\PY@tc##1{\textcolor[rgb]{0.40,0.40,0.40}{##1}}}
\expandafter\def\csname PY@tok@mh\endcsname{\def\PY@tc##1{\textcolor[rgb]{0.40,0.40,0.40}{##1}}}
\expandafter\def\csname PY@tok@go\endcsname{\def\PY@tc##1{\textcolor[rgb]{0.53,0.53,0.53}{##1}}}
\expandafter\def\csname PY@tok@ge\endcsname{\let\PY@it=\textit}
\expandafter\def\csname PY@tok@vc\endcsname{\def\PY@tc##1{\textcolor[rgb]{0.10,0.09,0.49}{##1}}}
\expandafter\def\csname PY@tok@il\endcsname{\def\PY@tc##1{\textcolor[rgb]{0.40,0.40,0.40}{##1}}}
\expandafter\def\csname PY@tok@cs\endcsname{\let\PY@it=\textit\def\PY@tc##1{\textcolor[rgb]{0.25,0.50,0.50}{##1}}}
\expandafter\def\csname PY@tok@cp\endcsname{\def\PY@tc##1{\textcolor[rgb]{0.74,0.48,0.00}{##1}}}
\expandafter\def\csname PY@tok@gi\endcsname{\def\PY@tc##1{\textcolor[rgb]{0.00,0.63,0.00}{##1}}}
\expandafter\def\csname PY@tok@gh\endcsname{\let\PY@bf=\textbf\def\PY@tc##1{\textcolor[rgb]{0.00,0.00,0.50}{##1}}}
\expandafter\def\csname PY@tok@ni\endcsname{\let\PY@bf=\textbf\def\PY@tc##1{\textcolor[rgb]{0.60,0.60,0.60}{##1}}}
\expandafter\def\csname PY@tok@nl\endcsname{\def\PY@tc##1{\textcolor[rgb]{0.63,0.63,0.00}{##1}}}
\expandafter\def\csname PY@tok@nn\endcsname{\let\PY@bf=\textbf\def\PY@tc##1{\textcolor[rgb]{0.00,0.00,1.00}{##1}}}
\expandafter\def\csname PY@tok@no\endcsname{\def\PY@tc##1{\textcolor[rgb]{0.53,0.00,0.00}{##1}}}
\expandafter\def\csname PY@tok@na\endcsname{\def\PY@tc##1{\textcolor[rgb]{0.49,0.56,0.16}{##1}}}
\expandafter\def\csname PY@tok@nb\endcsname{\def\PY@tc##1{\textcolor[rgb]{0.00,0.50,0.00}{##1}}}
\expandafter\def\csname PY@tok@nc\endcsname{\let\PY@bf=\textbf\def\PY@tc##1{\textcolor[rgb]{0.00,0.00,1.00}{##1}}}
\expandafter\def\csname PY@tok@nd\endcsname{\def\PY@tc##1{\textcolor[rgb]{0.67,0.13,1.00}{##1}}}
\expandafter\def\csname PY@tok@ne\endcsname{\let\PY@bf=\textbf\def\PY@tc##1{\textcolor[rgb]{0.82,0.25,0.23}{##1}}}
\expandafter\def\csname PY@tok@nf\endcsname{\def\PY@tc##1{\textcolor[rgb]{0.00,0.00,1.00}{##1}}}
\expandafter\def\csname PY@tok@si\endcsname{\let\PY@bf=\textbf\def\PY@tc##1{\textcolor[rgb]{0.73,0.40,0.53}{##1}}}
\expandafter\def\csname PY@tok@s2\endcsname{\def\PY@tc##1{\textcolor[rgb]{0.73,0.13,0.13}{##1}}}
\expandafter\def\csname PY@tok@vi\endcsname{\def\PY@tc##1{\textcolor[rgb]{0.10,0.09,0.49}{##1}}}
\expandafter\def\csname PY@tok@nt\endcsname{\let\PY@bf=\textbf\def\PY@tc##1{\textcolor[rgb]{0.00,0.50,0.00}{##1}}}
\expandafter\def\csname PY@tok@nv\endcsname{\def\PY@tc##1{\textcolor[rgb]{0.10,0.09,0.49}{##1}}}
\expandafter\def\csname PY@tok@s1\endcsname{\def\PY@tc##1{\textcolor[rgb]{0.73,0.13,0.13}{##1}}}
\expandafter\def\csname PY@tok@kd\endcsname{\let\PY@bf=\textbf\def\PY@tc##1{\textcolor[rgb]{0.00,0.50,0.00}{##1}}}
\expandafter\def\csname PY@tok@sh\endcsname{\def\PY@tc##1{\textcolor[rgb]{0.73,0.13,0.13}{##1}}}
\expandafter\def\csname PY@tok@sc\endcsname{\def\PY@tc##1{\textcolor[rgb]{0.73,0.13,0.13}{##1}}}
\expandafter\def\csname PY@tok@sx\endcsname{\def\PY@tc##1{\textcolor[rgb]{0.00,0.50,0.00}{##1}}}
\expandafter\def\csname PY@tok@bp\endcsname{\def\PY@tc##1{\textcolor[rgb]{0.00,0.50,0.00}{##1}}}
\expandafter\def\csname PY@tok@c1\endcsname{\let\PY@it=\textit\def\PY@tc##1{\textcolor[rgb]{0.25,0.50,0.50}{##1}}}
\expandafter\def\csname PY@tok@kc\endcsname{\let\PY@bf=\textbf\def\PY@tc##1{\textcolor[rgb]{0.00,0.50,0.00}{##1}}}
\expandafter\def\csname PY@tok@c\endcsname{\let\PY@it=\textit\def\PY@tc##1{\textcolor[rgb]{0.25,0.50,0.50}{##1}}}
\expandafter\def\csname PY@tok@mf\endcsname{\def\PY@tc##1{\textcolor[rgb]{0.40,0.40,0.40}{##1}}}
\expandafter\def\csname PY@tok@err\endcsname{\def\PY@bc##1{\setlength{\fboxsep}{0pt}\fcolorbox[rgb]{1.00,0.00,0.00}{1,1,1}{\strut ##1}}}
\expandafter\def\csname PY@tok@mb\endcsname{\def\PY@tc##1{\textcolor[rgb]{0.40,0.40,0.40}{##1}}}
\expandafter\def\csname PY@tok@ss\endcsname{\def\PY@tc##1{\textcolor[rgb]{0.10,0.09,0.49}{##1}}}
\expandafter\def\csname PY@tok@sr\endcsname{\def\PY@tc##1{\textcolor[rgb]{0.73,0.40,0.53}{##1}}}
\expandafter\def\csname PY@tok@mo\endcsname{\def\PY@tc##1{\textcolor[rgb]{0.40,0.40,0.40}{##1}}}
\expandafter\def\csname PY@tok@kn\endcsname{\let\PY@bf=\textbf\def\PY@tc##1{\textcolor[rgb]{0.00,0.50,0.00}{##1}}}
\expandafter\def\csname PY@tok@mi\endcsname{\def\PY@tc##1{\textcolor[rgb]{0.40,0.40,0.40}{##1}}}
\expandafter\def\csname PY@tok@gp\endcsname{\let\PY@bf=\textbf\def\PY@tc##1{\textcolor[rgb]{0.00,0.00,0.50}{##1}}}
\expandafter\def\csname PY@tok@o\endcsname{\def\PY@tc##1{\textcolor[rgb]{0.40,0.40,0.40}{##1}}}
\expandafter\def\csname PY@tok@kr\endcsname{\let\PY@bf=\textbf\def\PY@tc##1{\textcolor[rgb]{0.00,0.50,0.00}{##1}}}
\expandafter\def\csname PY@tok@s\endcsname{\def\PY@tc##1{\textcolor[rgb]{0.73,0.13,0.13}{##1}}}
\expandafter\def\csname PY@tok@kp\endcsname{\def\PY@tc##1{\textcolor[rgb]{0.00,0.50,0.00}{##1}}}
\expandafter\def\csname PY@tok@w\endcsname{\def\PY@tc##1{\textcolor[rgb]{0.73,0.73,0.73}{##1}}}
\expandafter\def\csname PY@tok@kt\endcsname{\def\PY@tc##1{\textcolor[rgb]{0.69,0.00,0.25}{##1}}}
\expandafter\def\csname PY@tok@ow\endcsname{\let\PY@bf=\textbf\def\PY@tc##1{\textcolor[rgb]{0.67,0.13,1.00}{##1}}}
\expandafter\def\csname PY@tok@sb\endcsname{\def\PY@tc##1{\textcolor[rgb]{0.73,0.13,0.13}{##1}}}
\expandafter\def\csname PY@tok@k\endcsname{\let\PY@bf=\textbf\def\PY@tc##1{\textcolor[rgb]{0.00,0.50,0.00}{##1}}}
\expandafter\def\csname PY@tok@se\endcsname{\let\PY@bf=\textbf\def\PY@tc##1{\textcolor[rgb]{0.73,0.40,0.13}{##1}}}
\expandafter\def\csname PY@tok@sd\endcsname{\let\PY@it=\textit\def\PY@tc##1{\textcolor[rgb]{0.73,0.13,0.13}{##1}}}

\def\PYZbs{\char`\\}
\def\PYZus{\char`\_}
\def\PYZob{\char`\{}
\def\PYZcb{\char`\}}
\def\PYZca{\char`\^}
\def\PYZam{\char`\&}
\def\PYZlt{\char`\<}
\def\PYZgt{\char`\>}
\def\PYZsh{\char`\#}
\def\PYZpc{\char`\%}
\def\PYZdl{\char`\$}
\def\PYZhy{\char`\-}
\def\PYZsq{\char`\'}
\def\PYZdq{\char`\"}
\def\PYZti{\char`\~}
% for compatibility with earlier versions
\def\PYZat{@}
\def\PYZlb{[}
\def\PYZrb{]}
\makeatother


    % Exact colors from NB
    \definecolor{incolor}{rgb}{0.0, 0.0, 0.5}
    \definecolor{outcolor}{rgb}{0.545, 0.0, 0.0}



    
    % Prevent overflowing lines due to hard-to-break entities
    \sloppy 
    % Setup hyperref package
    \hypersetup{
      breaklinks=true,  % so long urls are correctly broken across lines
      colorlinks=true,
      urlcolor=blue,
      linkcolor=darkorange,
      citecolor=darkgreen,
      }
    % Slightly bigger margins than the latex defaults
    
    \geometry{verbose,tmargin=1in,bmargin=1in,lmargin=1in,rmargin=1in}
    
    

    \begin{document}
    
    
    \maketitle
    
    

    
    \begin{Verbatim}[commandchars=\\\{\}]
{\color{incolor}In [{\color{incolor}1}]:} \PY{o}{\PYZpc{}}\PY{k}{matplotlib} inline
        \PY{k+kn}{from} \PY{n+nn}{\PYZus{}\PYZus{}future\PYZus{}\PYZus{}} \PY{k+kn}{import} \PY{n}{print\PYZus{}function}
        \PY{k+kn}{import} \PY{n+nn}{os}
        \PY{k+kn}{import} \PY{n+nn}{sys}
        \PY{k+kn}{import} \PY{n+nn}{numpy} \PY{k+kn}{as} \PY{n+nn}{np}
        \PY{k+kn}{import} \PY{n+nn}{matplotlib.pyplot} \PY{k+kn}{as} \PY{n+nn}{plt}
        \PY{k+kn}{import} \PY{n+nn}{plot\PYZus{}domain}
        \PY{n}{plot\PYZus{}domain}\PY{o}{.}\PY{n}{henry\PYZus{}domain}\PY{p}{(}\PY{p}{)}
        \PY{n}{plt}\PY{o}{.}\PY{n}{show}\PY{p}{(}\PY{p}{)}
\end{Verbatim}

    \begin{center}
    \adjustimage{max size={0.9\linewidth}{0.9\paperheight}}{errvarexample_files/errvarexample_0_0.png}
    \end{center}
    { \hspace*{\fill} \\}
    
    \subsection{Model background}\label{model-background}

Here is an example based on the Henry saltwater intrusion problem. The
synthetic model is a 2-dimensional SEAWAT model (X-Z domain) with 1 row,
120 columns and 20 layers. The left boundary is a specified flux of
freshwater, the right boundary is a specified head and concentration
saltwater boundary. The model has two stress periods: an initial steady
state (calibration) period, then a transient period with less flux
(forecast).

    The inverse problem has 603 parameters: 600 hydraulic conductivity pilot
points, 1 global hydraulic conductivity, 1 specified flux multiplier for
history matching and 1 specified flux multiplier for forecast
conditions. The inverse problem has 36 obseravtions (21 heads and 15
concentrations) measured at the end of the steady-state calibration
period. The forecasts of interest of the distance from the left model
edge to the 10\% seawater concentration in the basal model layer and the
concentration at location 10. Both of there forecasts are ``measured''
at the end of the forecast stress period. The forecasts are both in the
Jacobian matrix as zero-weight observations named \texttt{pd\_ten} and
\texttt{C\_obs10\_2}.I previously calculated the jacobian matrix, which
is in the \texttt{henry/} folder, along with the PEST control file.

Unlike the Schur's complement example notebook, here we will examine the
consequences of not adjusting the specified flux multiplier parameters
(\texttt{mult1} and \texttt{mult2}) during inversion, since these types
of model inputs are not typically considered for adjustment.

    \subsection{\texorpdfstring{Using
\texttt{pyemu}}{Using pyemu}}\label{using-pyemu}

    \begin{Verbatim}[commandchars=\\\{\}]
{\color{incolor}In [{\color{incolor}2}]:} \PY{k+kn}{import} \PY{n+nn}{pyemu}
\end{Verbatim}

    First create a linear\_analysis object. We will use \texttt{err\_var}
derived type, which replicates the behavior of the \texttt{PREDVAR}
suite of PEST as well as \texttt{ident\_par} utility. We pass it the
name of the jacobian matrix file. Since we don't pass an explicit
argument for \texttt{parcov} or \texttt{obscov}, \texttt{pyemu} attempts
to build them from the parameter bounds and observation weights in a
pest control file (.pst) with the same base case name as the jacobian.
Since we are interested in forecast uncertainty as well as parameter
uncertainty, we also pass the names of the forecast sensitivity vectors
we are interested in, which are stored in the jacobian as well. Note
that the \texttt{forecasts} argument can be a mixed list of observation
names, other jacobian files or PEST-compatible ASCII matrix files.
Remember you can pass a filename to the \texttt{verbose} argument to
write log file.

Since most groundwater model history-matching analyses focus on
adjusting hetergeneous hydraulic properties and not boundary condition
elements, let's identify the \texttt{mult1} and \texttt{mult2}
parameters as \texttt{omitted} in the error variance analysis. We can
conceptually think of this action as excluding the \texttt{mult1} and
\texttt{mult2} parameters from the history-matching process. Later we
will explicitly calculate the penalty for not adjusting this parameter.

    \begin{Verbatim}[commandchars=\\\{\}]
{\color{incolor}In [{\color{incolor}3}]:} \PY{n}{forecasts} \PY{o}{=} \PY{p}{[}\PY{l+s}{\PYZdq{}}\PY{l+s}{pd\PYZus{}ten}\PY{l+s}{\PYZdq{}}\PY{p}{,}\PY{l+s}{\PYZdq{}}\PY{l+s}{c\PYZus{}obs10\PYZus{}2}\PY{l+s}{\PYZdq{}}\PY{p}{]}
        \PY{n}{la} \PY{o}{=} \PY{n}{pyemu}\PY{o}{.}\PY{n}{errvar}\PY{p}{(}\PY{n}{jco}\PY{o}{=}\PY{n}{os}\PY{o}{.}\PY{n}{path}\PY{o}{.}\PY{n}{join}\PY{p}{(}\PY{l+s}{\PYZdq{}}\PY{l+s}{henry}\PY{l+s}{\PYZdq{}}\PY{p}{,} \PY{l+s}{\PYZdq{}}\PY{l+s}{pest.jcb}\PY{l+s}{\PYZdq{}}\PY{p}{)}\PY{p}{,}
                          \PY{n}{forecasts}\PY{o}{=}\PY{n}{forecasts}\PY{p}{,}
                          \PY{n}{omitted\PYZus{}parameters}\PY{o}{=}\PY{p}{[}\PY{l+s}{\PYZdq{}}\PY{l+s}{mult1}\PY{l+s}{\PYZdq{}}\PY{p}{,}\PY{l+s}{\PYZdq{}}\PY{l+s}{mult2}\PY{l+s}{\PYZdq{}}\PY{p}{]}\PY{p}{)}
        \PY{k}{print}\PY{p}{(}\PY{n}{la}\PY{o}{.}\PY{n}{jco}\PY{o}{.}\PY{n}{shape}\PY{p}{)} \PY{c}{\PYZsh{}without the omitted parameter or the prior info}
\end{Verbatim}

    \begin{Verbatim}[commandchars=\\\{\}]
(73, 601)
    \end{Verbatim}

    \section{Parameter identifiability}\label{parameter-identifiability}

The \texttt{errvar} dervied type exposes a method to get a
\texttt{pandas} dataframe of parameter identifiability information.
Recall that parameter identifiability is expressed as
\(d_i = \Sigma(\mathbf{V}_{1i})^2\), where \(d_i\) is the parameter
identifiability, which ranges from 0 (not identified by the data) to 1
(full identified by the data), and \(\mathbf{V}_1\) are the right
singular vectors corresonding to non-(numerically) zero singular values.
First let's look at the singular spectrum of
\(\mathbf{Q}^{\frac{1}{2}}\mathbf{J}\), where \(\mathbf{Q}\) is the
cofactor matrix and \(\mathbf{J}\) is the jacobian:

    \begin{Verbatim}[commandchars=\\\{\}]
{\color{incolor}In [{\color{incolor}4}]:} \PY{n}{s} \PY{o}{=} \PY{n}{la}\PY{o}{.}\PY{n}{qhalfx}\PY{o}{.}\PY{n}{s}
\end{Verbatim}

    \begin{Verbatim}[commandchars=\\\{\}]
{\color{incolor}In [{\color{incolor}5}]:} \PY{k+kn}{import} \PY{n+nn}{pylab} \PY{k+kn}{as} \PY{n+nn}{plt}
        \PY{n}{figure} \PY{o}{=} \PY{n}{plt}\PY{o}{.}\PY{n}{figure}\PY{p}{(}\PY{n}{figsize}\PY{o}{=}\PY{p}{(}\PY{l+m+mi}{10}\PY{p}{,} \PY{l+m+mi}{5}\PY{p}{)}\PY{p}{)}
        \PY{n}{ax} \PY{o}{=} \PY{n}{plt}\PY{o}{.}\PY{n}{subplot}\PY{p}{(}\PY{l+m+mi}{111}\PY{p}{)}
        \PY{n}{ax}\PY{o}{.}\PY{n}{plot}\PY{p}{(}\PY{n}{s}\PY{o}{.}\PY{n}{x}\PY{p}{)}
        \PY{n}{ax}\PY{o}{.}\PY{n}{set\PYZus{}title}\PY{p}{(}\PY{l+s}{\PYZdq{}}\PY{l+s}{singular spectrum}\PY{l+s}{\PYZdq{}}\PY{p}{)}
        \PY{n}{ax}\PY{o}{.}\PY{n}{set\PYZus{}ylabel}\PY{p}{(}\PY{l+s}{\PYZdq{}}\PY{l+s}{power}\PY{l+s}{\PYZdq{}}\PY{p}{)}
        \PY{n}{ax}\PY{o}{.}\PY{n}{set\PYZus{}xlabel}\PY{p}{(}\PY{l+s}{\PYZdq{}}\PY{l+s}{singular value}\PY{l+s}{\PYZdq{}}\PY{p}{)}
        \PY{n}{ax}\PY{o}{.}\PY{n}{set\PYZus{}xlim}\PY{p}{(}\PY{l+m+mi}{0}\PY{p}{,}\PY{l+m+mi}{20}\PY{p}{)}
        \PY{n}{plt}\PY{o}{.}\PY{n}{show}\PY{p}{(}\PY{p}{)}
\end{Verbatim}

    \begin{center}
    \adjustimage{max size={0.9\linewidth}{0.9\paperheight}}{errvarexample_files/errvarexample_9_0.png}
    \end{center}
    { \hspace*{\fill} \\}
    
    We see that the singluar spectrum decays rapidly (not uncommon) and that
we can really only support about 3 right singular vectors even though we
have 600+ parameters in the inverse problem.

Let's get the identifiability dataframe at 15 singular vectors:

    \begin{Verbatim}[commandchars=\\\{\}]
{\color{incolor}In [{\color{incolor}6}]:} \PY{n}{ident\PYZus{}df} \PY{o}{=} \PY{n}{la}\PY{o}{.}\PY{n}{get\PYZus{}identifiability\PYZus{}dataframe}\PY{p}{(}\PY{l+m+mi}{3}\PY{p}{)} \PY{c}{\PYZsh{} the method is passed the number of singular vectors to include in V\PYZus{}1}
        \PY{n}{ident\PYZus{}df}\PY{o}{.}\PY{n}{sort}\PY{p}{(}\PY{p}{)}\PY{o}{.}\PY{n}{iloc}\PY{p}{[}\PY{l+m+mi}{0}\PY{p}{:}\PY{l+m+mi}{10}\PY{p}{]}
\end{Verbatim}

            \begin{Verbatim}[commandchars=\\\{\}]
{\color{outcolor}Out[{\color{outcolor}6}]:}           right\_sing\_vec\_1  right\_sing\_vec\_2  right\_sing\_vec\_3         ident
        global\_k      9.966007e-01      1.084435e-03      2.834900e-04  9.979687e-01
        kr01c01       3.816852e-13      2.180892e-10      9.040267e-09  9.258738e-09
        kr01c02       1.350066e-10      3.028964e-08      2.157027e-07  2.461273e-07
        kr01c03       5.034623e-10      9.942224e-08      5.901479e-07  6.900736e-07
        kr01c04       3.504914e-09      7.431553e-07      5.734484e-06  6.481144e-06
        kr01c05       4.865283e-08      1.076199e-05      9.525053e-05  1.060612e-04
        kr01c06       1.994828e-07      4.468856e-05      4.137545e-04  4.586425e-04
        kr01c07       3.518241e-07      7.861029e-05      7.174784e-04  7.964405e-04
        kr01c08       3.663874e-07      8.183073e-05      7.459506e-04  8.281478e-04
        kr01c09       3.755248e-07      8.321196e-05      7.337729e-04  8.173604e-04
\end{Verbatim}
        
    Plot the indentifiability:

    We see that the \texttt{global\_k} parameter has a much higher
identifiability than any one of the 600 pilot points

    \section{Forecast error variance}\label{forecast-error-variance}

Now let's explore the error variance of the forecasts we are interested
in. We will use an extended version of the forecast error variance
equation:

\(\sigma_{s - \hat{s}}^2 = \underbrace{\textbf{y}_i^T({\bf{I}} - {\textbf{R}})\boldsymbol{\Sigma}_{{\boldsymbol{\theta}}_i}({\textbf{I}} - {\textbf{R}})^T\textbf{y}_i}_{1} + \underbrace{{\textbf{y}}_i^T{\bf{G}}\boldsymbol{\Sigma}_{\mathbf{\epsilon}}{\textbf{G}}^T{\textbf{y}}_i}_{2} + \underbrace{{\bf{p}}\boldsymbol{\Sigma}_{{\boldsymbol{\theta}}_o}{\bf{p}}^T}_{3}\)

Where term 1 is the null-space contribution, term 2 is the solution
space contribution and term 3 is the model error term (the penalty for
not adjusting uncertain parameters). Remember the \texttt{mult1} and
\texttt{mult2} parameters that we marked as omitted? The consequences of
that action can now be explicitly evaluated. See Moore and Doherty
(2005) and White and other (2014) for more explanation of these terms.
Note that if you don't have any \texttt{omitted\_parameters}, the only
terms 1 and 2 contribute to error variance

First we need to create a list (or numpy ndarray) of the singular values
we want to test. Since we have \(\lt40\) data, we only need to test up
to \(40\) singular values because that is where the action is:

    \begin{Verbatim}[commandchars=\\\{\}]
{\color{incolor}In [{\color{incolor}7}]:} \PY{n}{sing\PYZus{}vals} \PY{o}{=} \PY{n}{np}\PY{o}{.}\PY{n}{arange}\PY{p}{(}\PY{l+m+mi}{40}\PY{p}{)}
\end{Verbatim}

    The \texttt{errvar} derived type exposes a convience method to get a
multi-index pandas dataframe with each of the terms of the error
variance equation:

    \begin{Verbatim}[commandchars=\\\{\}]
{\color{incolor}In [{\color{incolor}8}]:} \PY{n}{errvar\PYZus{}df} \PY{o}{=} \PY{n}{la}\PY{o}{.}\PY{n}{get\PYZus{}errvar\PYZus{}dataframe}\PY{p}{(}\PY{n}{sing\PYZus{}vals}\PY{p}{)}
        \PY{n}{errvar\PYZus{}df}\PY{o}{.}\PY{n}{iloc}\PY{p}{[}\PY{l+m+mi}{0}\PY{p}{:}\PY{l+m+mi}{10}\PY{p}{]}
\end{Verbatim}

            \begin{Verbatim}[commandchars=\\\{\}]
{\color{outcolor}Out[{\color{outcolor}8}]:}       first              second               third          
          c\_obs10\_2    pd\_ten c\_obs10\_2    pd\_ten c\_obs10\_2    pd\_ten
        0  0.839807  3.767852  0.000000  0.000000  0.183547  0.898063
        1  0.390040  1.143221  0.000251  0.001890  0.290467  1.512740
        2  0.152808  0.916769  0.001810  0.003379  0.254220  1.418017
        3  0.152567  0.888371  0.001821  0.004660  0.253266  1.447760
        4  0.148153  0.867596  0.002512  0.007910  0.257929  1.477116
        5  0.120518  0.867447  0.008497  0.007942  0.270438  1.474536
        6  0.096935  0.860581  0.015565  0.010000  0.275265  1.467581
        7  0.095462  0.860334  0.016207  0.010108  0.274894  1.467982
        8  0.093688  0.858730  0.017148  0.010959  0.274024  1.465795
        9  0.093310  0.845932  0.017417  0.020061  0.274652  1.456169
\end{Verbatim}
        
    plot the error variance components for each forecast:

    \begin{Verbatim}[commandchars=\\\{\}]
{\color{incolor}In [{\color{incolor}9}]:} \PY{n}{fig} \PY{o}{=} \PY{n}{plt}\PY{o}{.}\PY{n}{figure}\PY{p}{(}\PY{n}{figsize}\PY{o}{=}\PY{p}{(}\PY{l+m+mi}{10}\PY{p}{,} \PY{l+m+mi}{10}\PY{p}{)}\PY{p}{)}
        \PY{n}{ax\PYZus{}1}\PY{p}{,} \PY{n}{ax\PYZus{}2}\PY{o}{=} \PY{n}{plt}\PY{o}{.}\PY{n}{subplot}\PY{p}{(}\PY{l+m+mi}{211}\PY{p}{)}\PY{p}{,} \PY{n}{plt}\PY{o}{.}\PY{n}{subplot}\PY{p}{(}\PY{l+m+mi}{212}\PY{p}{)}
        \PY{n}{axes} \PY{o}{=} \PY{p}{[}\PY{n}{ax\PYZus{}1}\PY{p}{,}\PY{n}{ax\PYZus{}2}\PY{p}{]}
        
        \PY{n}{colors} \PY{o}{=} \PY{p}{\PYZob{}}\PY{l+s}{\PYZdq{}}\PY{l+s}{first}\PY{l+s}{\PYZdq{}}\PY{p}{:} \PY{l+s}{\PYZsq{}}\PY{l+s}{g}\PY{l+s}{\PYZsq{}}\PY{p}{,} \PY{l+s}{\PYZdq{}}\PY{l+s}{second}\PY{l+s}{\PYZdq{}}\PY{p}{:} \PY{l+s}{\PYZsq{}}\PY{l+s}{b}\PY{l+s}{\PYZsq{}}\PY{p}{,} \PY{l+s}{\PYZdq{}}\PY{l+s}{third}\PY{l+s}{\PYZdq{}}\PY{p}{:} \PY{l+s}{\PYZsq{}}\PY{l+s}{c}\PY{l+s}{\PYZsq{}}\PY{p}{\PYZcb{}}
        \PY{n}{max\PYZus{}idx} \PY{o}{=} \PY{l+m+mi}{19}
        \PY{n}{idx} \PY{o}{=} \PY{n}{sing\PYZus{}vals}\PY{p}{[}\PY{p}{:}\PY{n}{max\PYZus{}idx}\PY{p}{]}
        \PY{k}{for} \PY{n}{ipred}\PY{p}{,} \PY{n}{pred} \PY{o+ow}{in} \PY{n+nb}{enumerate}\PY{p}{(}\PY{n}{forecasts}\PY{p}{)}\PY{p}{:}
            \PY{n}{pred} \PY{o}{=} \PY{n}{pred}\PY{o}{.}\PY{n}{lower}\PY{p}{(}\PY{p}{)}
            \PY{n}{ax} \PY{o}{=} \PY{n}{axes}\PY{p}{[}\PY{n}{ipred}\PY{p}{]}
            \PY{n}{ax}\PY{o}{.}\PY{n}{set\PYZus{}title}\PY{p}{(}\PY{n}{pred}\PY{p}{)}
            \PY{n}{first} \PY{o}{=} \PY{n}{errvar\PYZus{}df}\PY{p}{[}\PY{p}{(}\PY{l+s}{\PYZdq{}}\PY{l+s}{first}\PY{l+s}{\PYZdq{}}\PY{p}{,} \PY{n}{pred}\PY{p}{)}\PY{p}{]}\PY{p}{[}\PY{p}{:}\PY{n}{max\PYZus{}idx}\PY{p}{]}
            \PY{n}{second} \PY{o}{=} \PY{n}{errvar\PYZus{}df}\PY{p}{[}\PY{p}{(}\PY{l+s}{\PYZdq{}}\PY{l+s}{second}\PY{l+s}{\PYZdq{}}\PY{p}{,} \PY{n}{pred}\PY{p}{)}\PY{p}{]}\PY{p}{[}\PY{p}{:}\PY{n}{max\PYZus{}idx}\PY{p}{]}
            \PY{n}{third} \PY{o}{=} \PY{n}{errvar\PYZus{}df}\PY{p}{[}\PY{p}{(}\PY{l+s}{\PYZdq{}}\PY{l+s}{third}\PY{l+s}{\PYZdq{}}\PY{p}{,} \PY{n}{pred}\PY{p}{)}\PY{p}{]}\PY{p}{[}\PY{p}{:}\PY{n}{max\PYZus{}idx}\PY{p}{]}
            \PY{n}{ax}\PY{o}{.}\PY{n}{bar}\PY{p}{(}\PY{n}{idx}\PY{p}{,} \PY{n}{first}\PY{p}{,} \PY{n}{width}\PY{o}{=}\PY{l+m+mf}{1.0}\PY{p}{,} \PY{n}{edgecolor}\PY{o}{=}\PY{l+s}{\PYZdq{}}\PY{l+s}{none}\PY{l+s}{\PYZdq{}}\PY{p}{,} \PY{n}{facecolor}\PY{o}{=}\PY{n}{colors}\PY{p}{[}\PY{l+s}{\PYZdq{}}\PY{l+s}{first}\PY{l+s}{\PYZdq{}}\PY{p}{]}\PY{p}{,} \PY{n}{label}\PY{o}{=}\PY{l+s}{\PYZdq{}}\PY{l+s}{first}\PY{l+s}{\PYZdq{}}\PY{p}{,}\PY{n}{bottom}\PY{o}{=}\PY{l+m+mf}{0.0}\PY{p}{)}
            \PY{n}{ax}\PY{o}{.}\PY{n}{bar}\PY{p}{(}\PY{n}{idx}\PY{p}{,} \PY{n}{second}\PY{p}{,} \PY{n}{width}\PY{o}{=}\PY{l+m+mf}{1.0}\PY{p}{,} \PY{n}{edgecolor}\PY{o}{=}\PY{l+s}{\PYZdq{}}\PY{l+s}{none}\PY{l+s}{\PYZdq{}}\PY{p}{,} \PY{n}{facecolor}\PY{o}{=}\PY{n}{colors}\PY{p}{[}\PY{l+s}{\PYZdq{}}\PY{l+s}{second}\PY{l+s}{\PYZdq{}}\PY{p}{]}\PY{p}{,} \PY{n}{label}\PY{o}{=}\PY{l+s}{\PYZdq{}}\PY{l+s}{second}\PY{l+s}{\PYZdq{}}\PY{p}{,} \PY{n}{bottom}\PY{o}{=}\PY{n}{first}\PY{p}{)}
            \PY{n}{ax}\PY{o}{.}\PY{n}{bar}\PY{p}{(}\PY{n}{idx}\PY{p}{,} \PY{n}{third}\PY{p}{,} \PY{n}{width}\PY{o}{=}\PY{l+m+mf}{1.0}\PY{p}{,} \PY{n}{edgecolor}\PY{o}{=}\PY{l+s}{\PYZdq{}}\PY{l+s}{none}\PY{l+s}{\PYZdq{}}\PY{p}{,} \PY{n}{facecolor}\PY{o}{=}\PY{n}{colors}\PY{p}{[}\PY{l+s}{\PYZdq{}}\PY{l+s}{third}\PY{l+s}{\PYZdq{}}\PY{p}{]}\PY{p}{,} \PY{n}{label}\PY{o}{=}\PY{l+s}{\PYZdq{}}\PY{l+s}{third}\PY{l+s}{\PYZdq{}}\PY{p}{,} \PY{n}{bottom}\PY{o}{=}\PY{n}{second}\PY{o}{+}\PY{n}{first}\PY{p}{)}
            \PY{n}{ax}\PY{o}{.}\PY{n}{set\PYZus{}xlim}\PY{p}{(}\PY{o}{\PYZhy{}}\PY{l+m+mi}{1}\PY{p}{,}\PY{n}{max\PYZus{}idx}\PY{o}{+}\PY{l+m+mi}{1}\PY{p}{)}
            \PY{n}{ax}\PY{o}{.}\PY{n}{set\PYZus{}xticks}\PY{p}{(}\PY{n}{idx}\PY{o}{+}\PY{l+m+mf}{0.5}\PY{p}{)}
            \PY{n}{ax}\PY{o}{.}\PY{n}{set\PYZus{}xticklabels}\PY{p}{(}\PY{n}{idx}\PY{p}{)}
            \PY{k}{if} \PY{n}{ipred} \PY{o}{==} \PY{l+m+mi}{2}\PY{p}{:}
                \PY{n}{ax}\PY{o}{.}\PY{n}{set\PYZus{}xlabel}\PY{p}{(}\PY{l+s}{\PYZdq{}}\PY{l+s}{singular value}\PY{l+s}{\PYZdq{}}\PY{p}{)}
            
            \PY{n}{ax}\PY{o}{.}\PY{n}{set\PYZus{}ylabel}\PY{p}{(}\PY{l+s}{\PYZdq{}}\PY{l+s}{error variance}\PY{l+s}{\PYZdq{}}\PY{p}{)}
            \PY{n}{ax}\PY{o}{.}\PY{n}{legend}\PY{p}{(}\PY{n}{loc}\PY{o}{=}\PY{l+s}{\PYZdq{}}\PY{l+s}{upper right}\PY{l+s}{\PYZdq{}}\PY{p}{)}
        \PY{n}{plt}\PY{o}{.}\PY{n}{show}\PY{p}{(}\PY{p}{)}
\end{Verbatim}

    \begin{center}
    \adjustimage{max size={0.9\linewidth}{0.9\paperheight}}{errvarexample_files/errvarexample_19_0.png}
    \end{center}
    { \hspace*{\fill} \\}
    
    Here we see the trade off between getting a good fit to push down the
null-space (1st) term and the penalty for overfitting (the rise of the
solution space (2nd) term)). The sum of the first two terms in the
``appearent'' error variance (e.g.~the uncertainty that standard
analyses would yield) without considering the contribution from the
omitted parameters. You can verify this be checking prior uncertainty
from the Schur's complement notebook against the zero singular value
result using only terms 1 and 2.

We also see the added penalty for not adjusting the \texttt{mult1} and
\texttt{mult2} parameters (3rd term). The ability to forecast the
distance from the left edge of the model to the 10\% saltwater
concentration and the forecast the concentration at location 10 has been
compromised by not adjusting \texttt{mult1} and \texttt{mult2} during
calibration.

Let's check the \texttt{errvar} results against the results from
\texttt{schur}. This is simple with \texttt{pyemu}, we simply cast the
\texttt{errvar} type to a \texttt{schur} type:

    \begin{Verbatim}[commandchars=\\\{\}]
{\color{incolor}In [{\color{incolor}10}]:} \PY{n}{schur} \PY{o}{=} \PY{n}{la}\PY{o}{.}\PY{n}{get}\PY{p}{(}\PY{n}{astype}\PY{o}{=}\PY{n}{pyemu}\PY{o}{.}\PY{n}{schur}\PY{p}{)}
         \PY{n}{schur\PYZus{}prior} \PY{o}{=} \PY{n}{schur}\PY{o}{.}\PY{n}{prior\PYZus{}forecast}
         \PY{n}{schur\PYZus{}post} \PY{o}{=} \PY{n}{schur}\PY{o}{.}\PY{n}{posterior\PYZus{}forecast}
         \PY{k}{print}\PY{p}{(}\PY{l+s}{\PYZdq{}}\PY{l+s}{\PYZob{}0:10s\PYZcb{} \PYZob{}1:\PYZgt{}12s\PYZcb{} \PYZob{}2:\PYZgt{}12s\PYZcb{} \PYZob{}3:\PYZgt{}12s\PYZcb{} \PYZob{}4:\PYZgt{}12s\PYZcb{}}\PY{l+s}{\PYZdq{}}
               \PY{o}{.}\PY{n}{format}\PY{p}{(}\PY{l+s}{\PYZdq{}}\PY{l+s}{forecast}\PY{l+s}{\PYZdq{}}\PY{p}{,}\PY{l+s}{\PYZdq{}}\PY{l+s}{errvar prior}\PY{l+s}{\PYZdq{}}\PY{p}{,}\PY{l+s}{\PYZdq{}}\PY{l+s}{errvar min}\PY{l+s}{\PYZdq{}}\PY{p}{,}
                       \PY{l+s}{\PYZdq{}}\PY{l+s}{schur prior}\PY{l+s}{\PYZdq{}}\PY{p}{,} \PY{l+s}{\PYZdq{}}\PY{l+s}{schur post}\PY{l+s}{\PYZdq{}}\PY{p}{)}\PY{p}{)}
         \PY{k}{for} \PY{n}{ipred}\PY{p}{,} \PY{n}{pred} \PY{o+ow}{in} \PY{n+nb}{enumerate}\PY{p}{(}\PY{n}{forecasts}\PY{p}{)}\PY{p}{:}
             \PY{n}{first} \PY{o}{=} \PY{n}{errvar\PYZus{}df}\PY{p}{[}\PY{p}{(}\PY{l+s}{\PYZdq{}}\PY{l+s}{first}\PY{l+s}{\PYZdq{}}\PY{p}{,} \PY{n}{pred}\PY{p}{)}\PY{p}{]}\PY{p}{[}\PY{p}{:}\PY{n}{max\PYZus{}idx}\PY{p}{]}
             \PY{n}{second} \PY{o}{=} \PY{n}{errvar\PYZus{}df}\PY{p}{[}\PY{p}{(}\PY{l+s}{\PYZdq{}}\PY{l+s}{second}\PY{l+s}{\PYZdq{}}\PY{p}{,} \PY{n}{pred}\PY{p}{)}\PY{p}{]}\PY{p}{[}\PY{p}{:}\PY{n}{max\PYZus{}idx}\PY{p}{]}  
             \PY{n}{min\PYZus{}ev} \PY{o}{=} \PY{n}{np}\PY{o}{.}\PY{n}{min}\PY{p}{(}\PY{n}{first} \PY{o}{+} \PY{n}{second}\PY{p}{)}
             \PY{n}{prior\PYZus{}ev} \PY{o}{=} \PY{n}{first}\PY{p}{[}\PY{l+m+mi}{0}\PY{p}{]} \PY{o}{+} \PY{n}{second}\PY{p}{[}\PY{l+m+mi}{0}\PY{p}{]}
             \PY{n}{prior\PYZus{}sh} \PY{o}{=} \PY{n}{schur\PYZus{}prior}\PY{p}{[}\PY{n}{pred}\PY{p}{]}
             \PY{n}{post\PYZus{}sh} \PY{o}{=} \PY{n}{schur\PYZus{}post}\PY{p}{[}\PY{n}{pred}\PY{p}{]}
             \PY{k}{print}\PY{p}{(}\PY{l+s}{\PYZdq{}}\PY{l+s}{\PYZob{}0:12s\PYZcb{} \PYZob{}1:12.6f\PYZcb{} \PYZob{}2:12.6f\PYZcb{} \PYZob{}3:12.6\PYZcb{} \PYZob{}4:12.6f\PYZcb{}}\PY{l+s}{\PYZdq{}}
                   \PY{o}{.}\PY{n}{format}\PY{p}{(}\PY{n}{pred}\PY{p}{,}\PY{n}{prior\PYZus{}ev}\PY{p}{,}\PY{n}{min\PYZus{}ev}\PY{p}{,}\PY{n}{prior\PYZus{}sh}\PY{p}{,}\PY{n}{post\PYZus{}sh}\PY{p}{)}\PY{p}{)}
\end{Verbatim}

    \begin{Verbatim}[commandchars=\\\{\}]
forecast   errvar prior   errvar min  schur prior   schur post
pd\_ten           3.767852     0.865610      3.76785     0.832397
c\_obs10\_2        0.839807     0.108566     0.839807     0.099357
    \end{Verbatim}

    We see that the prior from \texttt{schur} class matches the two-term
\texttt{errvar} result at zero singular values. We also see, as
expected, the posterior from \texttt{schur} is slightly lower than the
two-term \texttt{errvar} result. This shows us that the ``appearent''
uncertainty in these predictions, as found through application of Bayes
equation, is being under estimated because if the ill effects of the
omitted \texttt{mult1} and \texttt{mult2} parameters.


    % Add a bibliography block to the postdoc
    
    
    
    \end{document}
